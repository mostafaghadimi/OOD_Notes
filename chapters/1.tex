\فصل{مروری بر نمودار \لر{UML}}

\قسمت {}در این فصل به مرور و آشنایی با نمودارهای زبان مدل‌سازی یک‌پارچه خواهیم پرداخت.




\قسمت{تعریف}

\لر{UML} یا زبان مدل‌سازی یک‌پارچه \زیرنویس{\لر{Unified Modeling Language}}، زبان استاندارد مهندسی نرم‌افزار برای مدل‌سازی است. مجموعه‌ای از بهترین رویه‌های عملی \زیرنویس{practice} را برای مدل کردن سیستم‌های بزرگ و پیچیده، ارائه می‌دهد.

\مهم{کاربرد:} برای مشخص کردن \زیرنویس{specifying}،  مجسم کردن\زیرنویس{visualizing}، ساختن \زیرنویس{constructing} و مستند کردن \زیرنویس{documenting} محصولات \زیرنویس{\لر{artifact} در این‌جا به معنی مصنوع و محصول است.} مهندسی نرم‌افزار و هم‌چنین مدل‌سازی کسب‌وکار از آن استفاده می‌شود.

\مهم{سوال:} در چه صورت می‌توان از \لر{UML} استفاده کرد؟ در صورتی که بتوان برای برای آن سیستم \لر{object} متصور د و رفتار سیستم را بتوان به‌صورت تعامل مجموعه‌ای از اشیاء نمایش داد. 

\قسمت{} این اسلاید، تاثیر زبان‌های مدل‌سازی (بعضی از مستطیل‌های طوسی مانند \لر{C++}) را بر روی \لر{UML} نمایش می‌دهد. 

\مهم{نکته ۱:} روش \لر{CRC} \زیرنویس{\لر{Class Responsibility Collaboration}} یکی از مهم‌ترین روش‌ها برای تشخیص کلاس است. 

\مهم{نکته ۲:} منظور از \لر{Formal Specification} تعریف‌های ریاضی است.

\مهم{نکته ۳:} تاثیرگذارترین متدولوژی‌ها روی کل‌گیری \لر{UML} عبارتند از: 
\شروع{شمارش}
\فقره \لر{Objectory}
\فقره \لر{Booch}
\فقره \لر{OMT}
\پایان{شمارش}

زیرا پایه‌گذاران این متدولوژی‌ها، \لر{UML} را نیز ابداع کرده‌اند.

\قسمت{انواع نمودارها}

انواع نمودارهای \لر{UML} به دو دسته‌ی زیر تقسیم می‌شوند \زیرنویس{در مهندسی نرم‌افزار تعداد دسته‌ها برابر با سه است و متعلق به نمودار‌های وظیفه‌ای یا \لر{functional} است.}:
\شروع{شمارش}
\فقره ساختاری یا \لر{Structural}: سیستم از چه اجزایی تشکیل شده است و چه رابطه‌ای با یکدیگر دارند؛ مانند نمودار کلاس. 
\فقره اجزای داخلی چگونه با یک‌دیگر کار می‌کنند و ترتیب انجام کارها به چه شکلی است به عبارت دیگر تقدم-تاخر وجود دارد؛ مانند نمودار \لر{flowchart}
\فقره 
\پایان{شمارش}